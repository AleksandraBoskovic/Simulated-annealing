% !TEX encoding = UTF-8 Unicode
\documentclass[a4paper]{article}

\usepackage{color}
\usepackage{url}
\usepackage[T2A]{fontenc} % enable Cyrillic fonts
\usepackage[utf8]{inputenc} % make weird characters work
\usepackage{graphicx}
\usepackage{amsmath}

\usepackage[english,serbian]{babel}
%\usepackage[english,serbianc]{babel} %ukljuciti babel sa ovim opcijama, umesto gornjim, ukoliko se koristi cirilica

\usepackage[unicode]{hyperref}
\hypersetup{colorlinks,citecolor=green,filecolor=green,linkcolor=blue,urlcolor=blue}

\usepackage{listings}

%\newtheorem{primer}{Пример}[section] %ćirilični primer
\newtheorem{primer}{Primer}[section]

\definecolor{mygreen}{rgb}{0,0.6,0}
\definecolor{mygray}{rgb}{0.5,0.5,0.5}
\definecolor{mymauve}{rgb}{0.58,0,0.82}

\lstset{ 
  backgroundcolor=\color{white},   % choose the background color; you must add \usepackage{color} or \usepackage{xcolor}; should come as last argument
  basicstyle=\scriptsize\ttfamily,        % the size of the fonts that are used for the code
  breakatwhitespace=false,         % sets if automatic breaks should only happen at whitespace
  breaklines=true,                 % sets automatic line breaking
  captionpos=b,                    % sets the caption-position to bottom
  commentstyle=\color{mygreen},    % comment style
  deletekeywords={...},            % if you want to delete keywords from the given language
  escapeinside={\%*}{*)},          % if you want to add LaTeX within your code
  extendedchars=true,              % lets you use non-ASCII characters; for 8-bits encodings only, does not work with UTF-8
  firstnumber=1000,                % start line enumeration with line 1000
  frame=single,	                   % adds a frame around the code
  keepspaces=true,                 % keeps spaces in text, useful for keeping indentation of code (possibly needs columns=flexible)
  keywordstyle=\color{blue},       % keyword style
  language=Python,                 % the language of the code
  morekeywords={*,...},            % if you want to add more keywords to the set
  numbers=left,                    % where to put the line-numbers; possible values are (none, left, right)
  numbersep=5pt,                   % how far the line-numbers are from the code
  numberstyle=\tiny\color{mygray}, % the style that is used for the line-numbers
  rulecolor=\color{black},         % if not set, the frame-color may be changed on line-breaks within not-black text (e.g. comments (green here))
  showspaces=false,                % show spaces everywhere adding particular underscores; it overrides 'showstringspaces'
  showstringspaces=false,          % underline spaces within strings only
  showtabs=false,                  % show tabs within strings adding particular underscores
  stepnumber=2,                    % the step between two line-numbers. If it's 1, each line will be numbered
  stringstyle=\color{mymauve},     % string literal style
  tabsize=2,	                   % sets default tabsize to 2 spaces
  title=\lstname                   % show the filename of files included with \lstinputlisting; also try caption instead of title
}

\begin{document}

\title{Naslov seminarskog rada\\ \small{Seminarski rad u okviru kursa\\Metodologija stručnog i naučnog rada\\ Matematički fakultet}}

\author{Aleksandra Bošković, Milena Stojić, treći autor, četvrti autor\\ kontakt email aleksandra94@hotmail.rs, mstojic39@yahoo.com, trećeg, četvrtog autora}

%\date{9.~april 2015.}

\maketitle

\abstract{
U ovom tekstu je ukratko prikazana osnovna forma seminarskog rada. Obratite pažnju da je pored ove .pdf datoteke, u prilogu i odgovarajuća .tex datoteka, kao i .bib datoteka korišćena za generisanje literature. Na prvoj strani seminarskog rada su naslov, apstrakt i sadržaj, i to sve mora da stane na prvu stranu! Kako bi Vaš seminarski zadovoljio standarde i očekivanja, koristite uputstva i materijale sa predavanja na temu pisanja seminarskih radova. Ovo je samo šablon koji se odnosi na fizički izgled seminarskog rada (šablon koji \emph{morate} da koristite!) kao i par tehničkih pomoćnih uputstava. Pročitajte tekst pažljivo jer on sadrži i važne informacije vezane za zahteve obima i karakteristika seminarskog rada.}

\tableofcontents

\newpage

\section{Uvod}
\label{sec:uvod}

Kada budete predavali seminarski rad, imenujete datoteke tako da sadrže redni broj teme, temu seminarskog rada, kao i prezimena članova grupe. Precizna uputstva na temu imenovnja će biti data na formi za predaju seminarskog rada. Predaja seminarskih radova biće isključivo preko veb forme, a NE slanjem mejla. Link na formu će biti dat u okviru obaveštenja na strani kursa. Vodite računa da prilikom predavanja seminarskog rada predate samo one fajlove koji su neophodni za ponovno generisanje pdf datoteke. To znači da pomoćne fajlove, kao što su .log, .out, .blg, .toc, .aux i slično, \textbf{ne treba predavati}.


\section{Algoritam}
Ideja implementacije algoritma simuliranog kaljenja je sledeća.U svakoj iteraciji algotitma primenjenog na diskretan problem poredimo dve vrednosti rešenja, trenutno najbolje i novo generisano rešenje. Zavisno od uslova vršimo odabir izmedju ova dva resenja koje prelazi u narednu iteraciju. Proces ponavljamo  dok god se ne zadovolji uslov zaustavljanja. Uslov zaustavljanja moze biti zadat konačnim brojem iteracija ili dok se ne dostigne trazeni maksimum(minimum).

\subsection{Definisanje termina}
Da bismo opisali specifičnosti algoritma simuliranog kaljenja uvedimo nekoliko pojmova koji ce nam biti potrebni za razumevanje samog algoritma.Neka je $\Omega$ prostor mogucih resenja,$f:\Omega \rightarrow \Re$ funkcija definisana nad prostorom $\Omega$. Cilj je pronaći  $w\in\Omega$ za koje funkcija $f$ doseze maksimum(minimum) problema na koji primenjujemo algoritam simuliranog kaljenja.Ukoliko postoji $w*\in\Omega$ za koje  vazi:$$f(w*)\geq f(w) , \forall w \in \Omega$$ tada je $w*$ trazeni maksimum.Da bi ovakvo $w*$ postojalo neophodan uslov je da funkcija $f$ bude ogranicena na prostoru $\Omega$.Definišimo i funkciju $N(w)$ pomocu koje cemo odredjivati susedna rešenja rešenju  $w\in\Omega$ u svakoj iteraciji algoritma pretrage novog rešenja. \par

\subsection{Odabir resenja}
Neka je $w\in\Omega$ inicijalno rešenje problema koje proizvoljno biramo iz skupa $\Omega$ i $w'\in N(w)$ proizvoljno odabrano susedno rešenje nekim predefinisanim pravilom ili slučajnim izborom. Ukoliko pogledamo uslov da resenje bude maksimum zaključujemo da nam je bolje ono rešenje koje ima vecu vrednost funkcije $f$.Ukoliko bismo u svakoj iteraciji birali rešenje koje ima veću vrednost funkcije $f$ algoritam bi se sveo na algoritam penjanja uzbrdo.Problem algoritma penjanja uzbrdo se javlja kada se naidje na lokalni maksimum i tu se dalja pretraga zaglavljuje jer svako susedno resenje nije bolje od njega.Jos jedna situacija u kojoj algoritam penjanja uzbrdo pokazuje slabosti je pojava platoa gde iz istog razloga pretraga ostane zaglavljena.Da bismo resili ovakve probleme algoritam simuliranog kaljenja za odluku koje od dva resenja ce uzeti za sledecu iteraciju bazira na verovatnoci $p$ prihvatljivosti da novo resenje bude uzeto za trenutno u narednoj iteraciji.
$$ p =
  \begin{cases}
     \quad \exp(f(w')-f(w)/t_k),  \quad f(w) > f(w'), \\
      \quad f(w) \leq f(w') \\
  \end{cases}
$$ \\


gde je $t_k$ temperaturni parametar iteracije k i n maksimali broj iteracija za koji vazi:
$$\forall k\leq n, \quad t_k > 0 \quad \lim_{k \to \infty}t_k=0 $$

\subsection{Imlementacija algoritma}
\textbf{Ulaz:} Inicijalno rešenje $w$,početna vrednost $t_k$ kao i vrednost $M_k$ koja predstavlja koliko puta ponavljati iteraciju sa datom vrednošću $t_k$
 \par
\textbf{Izlaz:} $w$ najbolje rešenje datog problema koje je algoritam uspeo da pronadje
\newline
Ponavljaj:
\begin{itemize}
\item[] $m=0$
\item[]  Ponavljaj:
\begin{itemize}
\item[] generiši novo rešenje  $w'\in N(w)$
\item[] izračunaj vrednost  $\Delta=f(w')-f(w)$
\item[] ako je $\Delta \geq 0$
\begin{itemize}
\item[] $w=w'$ 
\end{itemize}
\item[]inače
\begin{itemize}
\item[] sa verovatnoćom $p=\exp(-\Delta/t_k)$ važi $w=w'$ 
\end{itemize}
\item[]$m=m+1$
\end{itemize}
\item[]dok ne bude $ m=M_k$
\item[] $k=k+1$
\end{itemize}
Dok god nije zadovoljen uslov zaustavljanja


\section{Genetsko kaljenje}
Dve osnovne metaheuristike koje pripadaju široj grupi algoritama pretage su simulirano kaljenje i genetski algoritom.Sa idejom da se iskoriste prednosti obe metoda nastaje genetsko kaljenje.Tvorac koncepta je Kenneth Price koji je 1994 u svo clanku u casopisu "Dr.Bobbs Journal" prvi put izneo ideju o spajanju simuliranog kaljenja i genetskog algoritma. \par
Simulirano kaljenje je algoritam koji se uglavnom primenjuje kada postoji jedno optimalno resenje, dok se genetski primenjuje u slucaju postojanja vise optimalnih resenja.Ovu ideju koristimo u algoritmu genetskog kaljenja. \par
Genetski algoritam kroz iteracije evoluira ka boljim resenjima od pocetno zadatih resenja.U svakoj iteraciji neophodn je odabrati jedinke za ukrstanje i mutaciju.Algoritam genetskog kaljenja ideju simuliranog kaljenja koristi u procesu odabira jedinki.



\begin{primer}
Problem zaustavljanja (eng.~{\em halting problem}) je neodlučiv \cite{haltingproblem}.
\end{primer}

\begin{primer}
Za prevođenje programa napisanih u programskom jeziku C može se koristiti GCC kompajler \cite{gcc}.
\end{primer}

\begin{primer}
 Da bi se ispitivala ispravost softvera, najpre je potrebno precizno definisati njegovo ponašanje \cite{laski2009software}. 
\end{primer}

\subsection{Prednosti i nedostaci simuliranog kaljenja}

Prednost simuliranog kaljenja (u odnosu na klasican genetski algoritam) je dobra lokalna pretraga.(c) Ono ce uvek naci najbolje resenje u neposrednoj blizini. Medjutim to je u isto vreme i slabost, iako u teoriji ovaj algoritam moze da konvergira ka globalnom optimalnom resenju. \par
Ovaj nedostatak moze da se prevazidje sa vise zasebnih izvrsavanja. Svako sledece izvrsavanje pocinjemo sa novim pocetnim resenjem. Na kraju uzimamo najbolji finalni rezultat. U narednim iteracijama ne koristimo znanje dobijeno u prethodnim iteracijama. To je mana ovog pristupa. \par
Prednost genetskog algoritma u odnosu na simulirano kaljenje je i u visestrukosti resenja. Simuliranim kaljenjem mi samo poboljsavamo jedno resenje. Bolja je pretraga zasnovana na $m$ > 1 dobrih resenja.

\section{Zaključak}
\label{sec:zakljucak}

Ovde pišem zaključak. 
Ovde pišem zaključak. 
Ovde pišem zaključak. 
Ovde pišem zaključak. 
Ovde pišem zaključak. 
Ovde pišem zaključak. 
Ovde pišem zaključak. 
Ovde pišem zaključak. 
Ovde pišem zaključak. 
Ovde pišem zaključak. 
Ovde pišem zaključak. 
Ovde pišem zaključak. 


\addcontentsline{toc}{section}{Literatura}
\appendix
\bibliography{seminarski} 
\bibliographystyle{plain}

\appendix

\end{document}